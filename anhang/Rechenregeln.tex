\subsection{Rechenregeln}

\subsubsection{Potenzgesetze}

\begin{equation*}
    \begin{split}
        a^0 &= 1\\
        a^{-n} &= \frac{1}{a^n}\\
        a^m\cdot a^n &= a^{m+n}\\
        \frac{a^m}{a^n} &= a^{m-n}\\
        (a^m)^n &= a^{m\cdot n}\\
        (a\cdot b)^n &= a^n\cdot b^n\\
        \left(\frac{a}{b}\right)^n &= \frac{a^n}{b^n}
    \end{split}
\end{equation*}


\subsubsection{Logarithmusgesetze}
\label{logarithmusgesetze}

\begin{equation*}
    \begin{split}
        \ln(a\cdot b) &= \ln(a)+\ln(b)\\
        \ln(a^b) &= b\cdot\ln(a)\\
        \ln(e^a) &= a\\
        \ln\left(\frac{a}{b}\right) &= \ln(a)-\ln(b)\\
        \frac{\partial \ln(x)}{\partial x} &= \frac{1}{x}
    \end{split}
\end{equation*}


\subsubsection{Integration durch Substitution}
\label{substitution}

\begin{equation*}
    \int f(g(x))g'(x)dx = \int f(u)du
\end{equation*}

\begin{itemize}
    \item Teilausdruck \(u=g(x)\) substituieren
    \item Grenzen des Integrals anpassen (falls nötig) \(\rightarrow\) beiden Grenzen in \(u=g(x)\) einsetzen
    \item Ableitung von \(u\) bilden: \(\frac{du}{dx}\) und nach \(dx\) umstellen
    \item Integral umschreiben: \(\int f(u)\frac{du}{dx}dx = \int f(u)du\)
\end{itemize}


\subsubsection{Trigonometrische Funktionen}
\begin{equation*}
    \begin{split}
        \sin^2(x)+\cos^2(x) &= 1\\
        \tan(x) &= \frac{\sin(x)}{\cos(x)}\\
        \cot(x) &= \frac{1}{\tan(x)} = \frac{\cos(x)}{\sin(x)}\\
        \sin(2x) &= 2\sin(x)\cos(x)\\
        \cos(2x) &= \cos^2(x)-\sin^2(x) = 2\cos^2(x)-1 = 1-2\sin^2(x)\\
        \sin(x\pm y) &= \sin(x)\cos(y)\pm\cos(x)\sin(y)\\
        \cos(x\pm y) &= \cos(x)\cos(y)\mp\sin(x)\sin(y)\\
        \tan(x\pm y) &= \frac{\tan(x)\pm\tan(y)}{1\mp\tan(x)\tan(y)}
    \end{split}
\end{equation*}

\subsubsection{Skalarprodukt}
\label{skalarprodukt}

\begin{equation*}
    \left\langle a-b, a-b\right\rangle = \left\langle a, a-b\right\rangle - \left\langle b, a-b\right\rangle = \left\langle a, a\right\rangle - \left\langle a, b\right\rangle - \left\langle b, a\right\rangle + \left\langle b, b\right\rangle = \left\langle a, a\right\rangle - 2\left\langle a, b\right\rangle + \left\langle b, b\right\rangle
\end{equation*}

\begin{equation*}
    \left\langle \gamma a, \lambda b\right\rangle = \gamma\left\langle a, \lambda b\right\rangle = \gamma\lambda\left\langle a, b\right\rangle
\end{equation*}

\subsection{Definitionen}

Ein \textbf{Kern} (\(Ker(A)\)) existiert, wenn \(\det(A) = 0\).\\
Der Kern einer Matrix A ist die Lösungsmenge von \(A \cdot \vec{v} = \vec{0}\)\\
\(\rightarrow\) LGS=0 durch elem. Zeilenoperationen lösen.\\

Das \textbf{Bild} (\(Im(A)\)) einer Matrix gibt an, welche Menge an Vektoren als Lösungen auftreten können (vgl. Wertebereich bei Funktionen).\\
Das Bild einer Matrix A ist die Lösungsmenge von \(A \cdot \vec{v} = \vec{b}\)\\


Der \textbf{Rang} (\(rank(A)\)) einer Matrix A ist die Anzahl der linear unabhängigen Zeilen- bzw. Spaltenvektoren.\\
Rang = Anzahl der Nichtnullzeilen der Matrix in Zeilenstufenform.\\
\(\rightarrow\) A durch elem. Zeilenoperationen umformen.\\

Die \textbf{Länge} eines Vektors \(\vec{v}\) ist die Wurzel aus dem Skalarprodukt mit sich selbst.\\
\(\rightarrow \|\vec{v}\| = \sqrt{\vec{v} \cdot \vec{v}} = \sqrt{v_1^2 + v_2^2 + \dots + v_n^2}\)\\
\subsection{Definitionen}

Ein \textbf{Kern} (\(Ker(A)\)) existiert, wenn \(\det(A) = 0\).\\
Der Kern einer Matrix A ist die Lösungsmenge von \(A \cdot \vec{v} = \vec{0}\) \(\rightarrow\) LGS=0 durch elem. Zeilenoperationen lösen.\\

Das \textbf{Bild} (\(Im(A)\)) einer Matrix gibt an, welche Menge an Vektoren als Lösungen auftreten können (vgl. Wertebereich bei Funktionen). Das Bild einer Matrix A ist die Lösungsmenge von \(A \cdot \vec{v} = \vec{b}\)\\


Der \textbf{Rang} (\(rank(A)\)) einer Matrix A ist die Anzahl der linear unabhängigen Spaltenvektoren\\
\underline{Ermittlung}: Spalten von links nach rechts \(\rightarrow\) ist die \(\text{Spalte}_i\) linear abhängig von den vorherigen? Rang = Anzahl der linear unabhängigen Spaltenvektoren\\
Verwendung z.B. zur Komprimierung von A:

\begin{equation*}
    A = \begin{pmatrix}
        \textbf{1} & \textbf{1} & 2 & 4 & 2\\
        \textbf{2} & \textbf{1} & 3 & 5 & 4\\
        \textbf{1} & \textbf{1} & 2 & 4 & 2\\
        \textbf{0} & \textbf{1} & 1 & 3 & 0\\        
    \end{pmatrix} \rightarrow \begin{array}{c}
        a_1 = 1 \cdot a_1 + 0 \cdot a_2\\
        a_2 = 0 \cdot a_1 + 1 \cdot a_2\\
        a_3 = 1 \cdot a_1 + 1 \cdot a_2\\
        a_4 = 1 \cdot a_1 + 3 \cdot a_2\\
        a_5 = 2 \cdot a_1 + 0 \cdot a_2\\
    \end{array} \rightarrow A = \begin{pmatrix}
        \textbf{1} & \textbf{1} \\
        \textbf{2} & \textbf{1} \\
        \textbf{1} & \textbf{1} \\
        \textbf{0} & \textbf{1} \\
    \end{pmatrix} \begin{pmatrix}
        1 & 0 & 1 & 1 & 2\\
        0 & 1 & 1 & 3 & 0
    \end{pmatrix}
\end{equation*}\\

Die \textbf{Länge} eines Vektors \(\vec{v}\) ist die Wurzel aus dem Skalarprodukt mit sich selbst.\\
\(\rightarrow \|\vec{v}\| = \sqrt{\left\langle v, v\right\rangle} = \sqrt{v_1^2 + v_2^2 + \dots + v_n^2}\)\\

Das \textbf{Skalarprodukt} \(\left\langle x, y\right\rangle \) zweier Vektoren ist die Summe der Produkte der jeweiligen Komponenten\\
\(\rightarrow x_1y_1+\cdots + x_dy_d\); man kann damit den von \(x\) und \(y\) eingeschlossenen Winkel als Zahl \(\theta \in [0,\pi]\) berechnen mit: \(\cos \theta = \frac{\left\langle x, y\right\rangle}{\|x\|\cdot \|y\|} \rightarrow\) Zwei Vektoren stehen senkrecht zueinander, wenn \(\left\langle x, y\right\rangle = 0\)\\
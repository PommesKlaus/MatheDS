\subsection{Determinante}

Spezielle Funktion, die einer \underline{quadratischen} Matrix eine Zahl zuordnet. Diese gibt an, wie sich das Volumen bei der durch die Matrix beschriebenen linearen Abbildung ändert. \\

\begin{itemize}
    \item \(det(A)= 0\rightarrow\) Matrix \(A\) ist nicht invertierbar; ist z.B. der Fall, wenn
    \begin{itemize}
        \item eine Zeile oder Spalte nur aus Nullen besteht
        \item 2 Zeilen/Spalten identisch sind
        \item Zeilen/Spalten ein Vielfaches einer anderen Zeile/Spalte sind.
    \end{itemize}
    \item \(det(I) = 1\)
    \item \(det(A) = det(A^T)\)
    \item \(det\begin{pmatrix}
            \lambda \cdot a & \lambda \cdot b \\
            c & d
        \end{pmatrix} = \lambda \cdot det\begin{pmatrix}
            a & b \\
            c & d
        \end{pmatrix} \rightarrow\) Eine Zeile mit \( \lambda \) multiplizieren
    \item \(det(\lambda\cdot A) = \lambda^n \cdot det(A)\) \(\rightarrow\) Ganze Matrix (\(\mathbb{R}^{nxn}\)) mit \( \lambda \) multiplizieren
    \item \(det(A^{-1}) = det(A)^{-1}=\frac{1}{det(A)}\)
    \item \(det(A\cdot B) = det(A)\cdot det(B)\)
    \item \(det\begin{pmatrix}
            a & b \\
            c & d
        \end{pmatrix} = -det\begin{pmatrix}
            c & d \\
            a & b
        \end{pmatrix}\rightarrow\) Zeilentausch: Vorzeichenwechsel
\end{itemize}



\begin{figure}[H]
    \begin{equation*}
        \det\begin{pmatrix}
        a & b \\
        c & d
        \end{pmatrix}
        = ad - bc
    \end{equation*}
\end{figure}

\begin{figure}[H]
    \begin{equation*}
        \det\begin{pmatrix}
        a & b & c \\
        d & e & f \\
        g & h & i
        \end{pmatrix}
        = aei + bfg + cdh - gec - hfa - ibd
    \end{equation*}
\end{figure}

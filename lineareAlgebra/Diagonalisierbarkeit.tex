\subsection{Diagonalisierbarkeit}

\subsubsection{Diagonalisierbarkeit}
\(A\) ist diagonalisierbar, wenn
\begin{itemize}
    \item für jeden Eigenwert von \(A\) die \underline{algebraische} Vielfachheit \textbf{gleich} der \underline{geometrischen} Vielfachheit ist, oder
    \item wenn alle Eigenwerte (\(\lambda_i\)) von \(A\) unterschiedlich sind.
\end{itemize}

Um die Diagonalmatrix \(D = S^{-1}AS\) bzw. \(A=SDS^{-1}\) zu bestimmen:
\begin{enumerate}
    \item Eigenwerte \(\lambda_i\) von \(A\) bestimmen \(\rightarrow\) \textit{Nullstellen char. Polynom}
    \item Eigenvektoren \(\vec{v_i}\) zu \(\lambda_i\) bestimmen \(\rightarrow\) \textit{Spalten der Matrix S}
    \item Diagonalmatrix \(D = diag(\lambda_1, \lambda_2, \hdots \lambda_i)\) bestimmen
\end{enumerate}

\subsubsection{Orthogonale Diagonalisierbarkeit}

Eine Matrix \(A \in \mathbb{R}^{n \times n}\) heißt orthogonal diagonalisierbar, falls es eine orthogonale Matrix \(S \in \mathbb{R}^{n \times n}\) gibt, so dass \(D = S^T A S = S^{-1}AS\) eine Diagonalmatrix ist (\(S^T S = I \Rightarrow \) Orthogonalität \(S^{-1} = S^T\)).\\

Dies ist genau dann der Fall, wenn \(A\) \underline{symmetrisch}  ist:
\begin{equation*}
    \boldsymbol{A^T} = (S D S^T)^T = (S^T)^T D^T S^T = SDS^T = \boldsymbol{A}
\end{equation*}

Vorgehensweise analog zur Diagonalisierbarkeit; zusätzlich müssen die Eigenvektoren \(\vec{v_i}\) zu \(\lambda_i\) noch normiert werden (\(\tilde{v_i}=\frac{v_i}{||v_i||} \))

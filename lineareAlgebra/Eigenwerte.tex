
\subsection{Eigenwerte, Eigenvektoren und Eigenraum}

Eine Zahl \(\lambda\) heißt Eigenwert der Matrix A, wenn es einen Vektor \(\vec{v}\) gibt, der nicht der Nullvektor ist, so dass gilt:

\begin{equation*}
    \begin{split}
        A v &= \lambda v \\
        A v - \lambda v &= 0 \\
        (A - \lambda I) v &= 0
    \end{split}
\end{equation*}


\subsubsection{Charakteristisches Polynom berechnen}
\label{sec:charPolynom}
Anstatt o.g. Gleichungzu lösen: Bestimmung der Nullstellen des charakteristischen Polynoms \(p_A(\lambda)\) der Matrix A.

\begin{equation*}
    \begin{split}
    p_A(\lambda) & = \det(A - \lambda I) \\
    & = \begin{vmatrix}
    a_{11} - \lambda & \cdots & a_{1n} \\
    \vdots & \ddots & \vdots \\
    a_{n1} & \cdots & a_{nn} - \lambda
    \end{vmatrix} \overset{!}{=} 0
    \end{split}
\end{equation*}

\subsubsection{Eigenvektoren berechnen}
Der zu einem Eigenwert \(\lambda_i\) gehörende Eigenvektor \(\vec{v_i}\) ist die Lösung der Gleichung:

\begin{equation*}
    \begin{split}
        A\vec{v_i} &= \lambda_i \vec{v_i} \\
        (A - \lambda_i I) \cdot \vec{v_i} &= \vec{0}
    \end{split}
\end{equation*}

\textit{Rechenweg:}

\begin{enumerate}
    \item \(\lambda_i\) für \(\lambda\) in die Matrix \((A-\lambda E)\) einsetzen (siehe charakterisches Polynom)
    \item Das folgende LGS durch elementare Zeilenoperationen lösen:\\
        \begin{equation*}
            \left( 
            \begin{array}{ccc|c}
                a_{11} - \lambda & \cdots & a_{1n} & 0\\
                \vdots & \ddots & \vdots & 0\\
                a_{n1} & \cdots & a_{nn} - \lambda & 0
            \end{array}
            \right)
        \end{equation*}
    \item Für Nullzeilen ergeben sich beliebige Lösungen, die gleich 1 gesetzt werden können.
\end{enumerate}


\subsubsection{Eigenraum berechnen}

Der Eigenraum \(E_A(\lambda_i)\) einer Matrix A zu einem Eigenwert \(\lambda_i\) ist die Menge aller Eigenvektoren \(\vec{v_i}\) zu \(\lambda_i\).\\

\textit{Lösung:}
Vielfaches der Eigenvektoren in Mengenschreibweise festhalten:\\

\begin{equation*}
    E_A(\lambda_i) = \{ k \cdot \vec{v_i} | k \in \mathbb{R} \}   
\end{equation*}

\subsubsection{algebraische vs. geometrische Vielfachheit von \(\lambda\)}
\begin{itemize}
    \item \textbf{algebraische Vielfachheit}: Anzahl gleicher Eigenwerte im charakteristischen Polynom
    \item \textbf{geometrische Vielfachheit}: Dimension (Anzahl der Vektoren) des Eigenraums \(E(\lambda)\); \(\leq\) algebraische Vielfachheit
\end{itemize}
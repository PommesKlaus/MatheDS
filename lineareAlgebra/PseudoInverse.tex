\subsection{Pseudo-Inverse \(A^+\)}

Approximation einer inversen Matrix Für nicht-quadratische Matrizen mit Hilfe der Singulärwertzerlegung (siehe \ref{SVD}).

\begin{equation*}
    A^+ = V \cdot \Sigma^{-1} \cdot U^T
\end{equation*}
wobei \(\Sigma^{-1}=diag(\sigma_1^{-1}, \hdots \sigma_r^{-1})\)\\

\textit{Eigenschaften:}
\begin{itemize}
    \item \(A  A^+  A = A\)
    \item \(A^+  A  A^+ = A^+\)
    \item \((A  A^+)^T = A  A^+\) \(\rightarrow\) \(A  A^+\) ist symmetrisch
    \item \((A^+  A)^T = A^+  A\) \(\rightarrow\) \(A^+  A\) ist symmetrisch
    \item \(A^+ = A^{-1}\), wenn A invertierbar ist
    \item \(A = U \Sigma V^T \Leftrightarrow A^T = V \Sigma U^T \)
\end{itemize}
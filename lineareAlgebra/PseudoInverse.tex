\subsection{Inverse Matrix \(A^{-1}\) für quadratische Matrizen}

\textit{Eigenschaften:} Siehe Eigenschaften unter \ref{PseudoInverse}; zusätzlich:
\begin{itemize}
    \item \(A^T = A^{-1}\) \(\rightarrow\) \(A\) ist orthogonal
    \item \(A  A^T = I\) \(\rightarrow\) \(A\) ist orthogonal
    \item \(A^T  A = I\) \(\rightarrow\) \(A\) ist orthogonal\\
\end{itemize}

Berechnung für eine \textbf{2x2-Matrix}:
\begin{equation*}
    A = \begin{pmatrix}
        a & b\\
        c & d
    \end{pmatrix} \rightarrow A^{-1} = \frac{1}{ad-bc} \begin{pmatrix}
        d & -b\\
        -c & a
    \end{pmatrix}
\end{equation*}

Berechnung für eine \textbf{3x3-Matrix}:
\begin{equation*}
    A = \begin{pmatrix}
        a & b & c\\
        d & e & f\\
        g & h & i
    \end{pmatrix} \rightarrow A^{-1} = \frac{1}{det(A)} \begin{pmatrix}
        ei-fh & ch-bi & bf-ce\\
        fg-di & ai-cg & cd-af\\
        dh-ge & bg-ah & ae-bd
    \end{pmatrix}
\end{equation*}


\subsection{Pseudo-Inverse \(A^+\)}
\label{PseudoInverse}

Approximation einer inversen Matrix für nicht-quadratische Matrizen mit Hilfe der Singulärwertzerlegung (siehe \ref{SVD}).

\begin{equation*}
    A^+ = V \cdot \Sigma^{-1} \cdot U^T
\end{equation*}
wobei \(\Sigma^{-1}=diag(\sigma_1^{-1}, \hdots \sigma_r^{-1})\)\\

\textit{Eigenschaften:}
\begin{itemize}
    \item \(A  A^+  A = A\)
    \item \(A^+  A  A^+ = A^+\)
    \item \((A  A^+)^T = A  A^+\) \(\rightarrow\) \(A  A^+\) ist symmetrisch
    \item \((A^+  A)^T = A^+  A\) \(\rightarrow\) \(A^+  A\) ist symmetrisch
    \item \(A^+ = A^{-1}\), wenn A invertierbar ist
    \item \(A = U \Sigma V^T \Leftrightarrow A^T = V \Sigma U^T \)
    \item \(V^TV=VV^T=I\) und \(U^TU=UU^T=I\)
\end{itemize}
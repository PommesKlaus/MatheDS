\section{Markov-Ketten}
	
\begin{center}
	\begin{tikzpicture}[->, >=stealth', auto, semithick, node distance=3cm]
	\tikzstyle{every state}=[fill=white,draw=black,thick,text=black,scale=1]
	\node[state]    (A)                     {$0$};
	\node[state]    (B)[right of=A]   {$1$};
	\node[state]    (C)[right of=B]   {$2$};
	\node[state]    (D)[right of=C]   {$3$};
	\path
	(A) edge[loop left]			node{$1$}	(A)
	(B) edge[bend left,below]	node{$1/3$}	(A)
	edge[bend left,above]		node{$2/3$}	(C)
	(C) edge[bend left,below]	node{$2/3$}	(B)
	edge[bend left,above]		node{$1/3$}	(D)
	(D) edge[loop right]		node{$1$}	(D);
	%\node[above=0.5cm] (A){Patch G};
	%\draw[red] ($(D)+(-1.5,0)$) ellipse (2cm and 3.5cm)node[yshift=3cm]{Patch H};
	\end{tikzpicture}
\end{center}

\emph{Eine homogene, \textbf{irreduzible}, \textbf{aperiodische} Markov-Kette mit endlichem Zustandsraum ist immer \textbf{stationär}, d.h. sie konvergiert gegen ihr statistisches Gleichgewicht.}

\subsection{Übergangsmatrix}

\begin{itemize}
	\item Zeilen: Quelle, von diesem Knoten startet man
	\item Spalten: Zielknoten
	\item \(p_{ij}\) gibt dann die Übergangswahrscheinlichkeiten an
	\begin{itemize}
		\item \(p_{ij} \in [0;1] \rightarrow\) Es sind nur Werte zwischen 0 und 1 erlaubt
		\item \(\sum_{j=1}^{n} p_{ij} = 1 \rightarrow\) Die Summe der Zeilenwahrscheinlichkeiten \(=1\)
	\end{itemize}
\end{itemize}

\subsection{Stationäre Verteilung}
Ein Zustandsvektor \(\pi\) heißt \emph{stationäre Verteilung} einer Markov-Kette, wenn gilt:
\begin{equation*}
    \pi \cdot P = \pi
\end{equation*}

\(\lim_{n\rightarrow\infty} P^n\) konvergiert zur stationären Verteilung \(\pi\), wenn die Markov-Kette homogen, irreduzibel und aperiodisch ist.\\

Gleichungen lösen, ggf. mit Hilfe von Parametern \(t\), wenn es keine eindeutige Lösung gibt. Wert für \(t\) bestimmen, indem die Summe der Komponenten des Vektors \(\pi\) gleich 1 gesetzt wird.\\

\textbf{Beispiel:}
\begin{equation*}
    P = \begin{pmatrix}
        0.5 & 0.5\\
        0.3 & 0.7
    \end{pmatrix}
\end{equation*}
\begin{equation*}
    \begin{pmatrix}
        \pi_1 & \pi_2
    \end{pmatrix} \cdot \begin{pmatrix}
        0.5 & 0.5\\
        0.3 & 0.7
    \end{pmatrix} = \begin{pmatrix}
        \pi_1 & \pi_2
    \end{pmatrix}
\end{equation*}

Da \(\sum \pi_i=1\) gilt

\begin{equation*}
    \pi_1 + \pi_2 = 1 \Leftrightarrow \underline{\pi_1 = 1 - \pi_2} \Leftrightarrow \underline{\pi_2 = 1 - \pi_1}
\end{equation*}

\(\pi_1\) und \(\pi_2\) in die folgenden Gleichungen einsetzen:

\begin{equation*}
    \pi_1 \cdot 0.5 + \pi_2 \cdot 0.3 = \pi_1 \Longrightarrow \pi_2 = \frac{5}{8}
\end{equation*}
\begin{equation*}
    \pi_1 \cdot 0.5 + \pi_2 \cdot 0.7 = \pi_2 \Longrightarrow \pi_1 = \frac{3}{8}
\end{equation*}

Daraus folgt der stationäre Zustandsvektor: \(\pi = \begin{pmatrix}
    \frac{3}{8} & \frac{5}{8}
\end{pmatrix}\)



\subsection{Irreduzibilität}

Es ist von jedem Zustand aus möglich, jeden anderen Zustand zu erreichen.
Die Prüfung kann manuell erfolgen.\\

\textbf{Beispiel:}
\(
    P = \begin{pmatrix}
        1-p & p/2 & p/2\\
        1 & 0 & 0\\
        1 & 0 & 0
    \end{pmatrix}
\) ist irreduzibel für \(p\in (0,1]\), da für \(p=0\) Zustände 2 und 3 nicht mehr erreichbar sind.\\

Gegenteil = \emph{reduzibel}: Man kann die große Markovkette in kleinere Teile zerlegen (reduzieren), die nicht miteinander verbunden sind.\\

\subsection{Aperiodizität}

Die Periode eines Zustands ist die größte gemeinsame Teiler aller Pfade, die zu diesem Zustand zurück führen.\\

Starte in einem Zustand \(i\) und gehe in einen Zustand \(j\). Die Periode von \(i\) ist die größte gemeinsame Teiler aller Pfade, die von \(j\) nach \(i\) führen.
Wenn die Periode von jedem Zustand \(i\) gleich 1 ist, ist die Markov-Kette aperiodisch. Das heißt:

\begin{equation*}
    d(z)=ggT\{n \in \mathbb{N} | P_{ii}^{(n)} > 0\} = 1
\end{equation*}

\textbf{Beispiel:}
\(
    P = \begin{pmatrix}
        0 & 1\\
        1 & 0
    \end{pmatrix}
\) ist \emph{nicht} aperiodisch, da \(d(z) = 2\); man springt immer zwischen den beiden Zuständen mit einer geraden Anzahl hin und her.\\

\begin{center}
	\begin{tikzpicture}[->, >=stealth', auto, semithick, node distance=3cm]
	\tikzstyle{every state}=[fill=white,draw=black,thick,text=black,scale=1]
	\node[state]    (A)               {$A$};
	\node[state]    (B)[right of=A]   {$B$};
	\path
	(A) edge[bend left]			node{$1$}	(B)
	(B) edge[bend left,below]	node{$1$}	(A);
	%\node[above=0.5cm] (A){Patch G};
	%\draw[red] ($(D)+(-1.5,0)$) ellipse (2cm and 3.5cm)node[yshift=3cm]{Patch H};
	\end{tikzpicture}
\end{center}

\textbf{Beispiel:}
\(
    P = \begin{pmatrix}
        1-p & p/2 & p/2\\
        1 & 0 & 0\\
        1 & 0 & 0
    \end{pmatrix}
\) ist aperiodisch für \(p\in [0,1)\), da bei \(p=1\) immer von Zustand 1 in 2 oder 3 gesprungen wird und wieder zurück.
Für alle anderen Werte wird in zwei Fällen auch hin- und zurückgesprungen. Allerdings gibt es auch die Möglichkeit, dass vom
Zustand \(A\) nicht geswechelt wird und man dort bleibt.\\

\begin{center}
	\begin{tikzpicture}[->, >=stealth', auto, semithick, node distance=3cm]
	\tikzstyle{every state}=[fill=white,draw=black,thick,text=black,scale=1]
	\node[state]    (A)               {$A$};
	\node[state]    (B)[right of=A]   {$B$};
	\node[state]    (C)[left of=A]    {$C$};
	% \node[state]    (D)[right of=C]   {$3$};
	\path
	(A) edge[loop left]			node{$1-p$}	(A)
    (A) edge[bend left]			node{$p/2$}	(B)
	(B) edge[bend left,below]	node{$1$}	(A)
    (A) edge[bend left, below]			node{$p/2$}	(C)
    (C) edge[bend left,below]	node{$1$}	(A);
	%\node[above=0.5cm] (A){Patch G};
	%\draw[red] ($(D)+(-1.5,0)$) ellipse (2cm and 3.5cm)node[yshift=3cm]{Patch H};
	\end{tikzpicture}
\end{center}

\subsection{Stoppzeiten}

Eine Stoppzeit ist eine Zufallsvariable, die das Eintreten eines Ereignisses beschreibt, das von der bisherigen Entwicklung eines stochastischen Prozesses abhängt.\\
\section{Markov-Ketten}
	
\begin{center}
	\begin{tikzpicture}[->, >=stealth', auto, semithick, node distance=3cm]
	\tikzstyle{every state}=[fill=white,draw=black,thick,text=black,scale=1]
	\node[state]    (A)                     {$0$};
	\node[state]    (B)[right of=A]   {$1$};
	\node[state]    (C)[right of=B]   {$2$};
	\node[state]    (D)[right of=C]   {$3$};
	\path
	(A) edge[loop left]			node{$1$}	(A)
	(B) edge[bend left,below]	node{$1/3$}	(A)
	edge[bend left,above]		node{$2/3$}	(C)
	(C) edge[bend left,below]	node{$2/3$}	(B)
	edge[bend left,above]		node{$1/3$}	(D)
	(D) edge[loop right]		node{$1$}	(D);
	%\node[above=0.5cm] (A){Patch G};
	%\draw[red] ($(D)+(-1.5,0)$) ellipse (2cm and 3.5cm)node[yshift=3cm]{Patch H};
	\end{tikzpicture}
\end{center}

\subsection{Übergangsmatrix}

\subsection{Stationäre Verteilung}
Ein Zustandsvektor \(\pi\) heißt \emph{stationäre Verteilung} einer Markov-Kette, wenn gilt:
\begin{equation*}
    \pi \cdot P = \pi
\end{equation*}

Gleichungen lösen, ggf. mit Hilfe von Parametern \(t\), wenn es keine eindeutige Lösung gibt. Wert für \(t\) bestimmen, indem die Summe der Komponenten des Vektors \(\pi\) gleich 1 gesetzt wird.\\

\textbf{Beispiel:}
\begin{equation*}
    P = \begin{pmatrix}
        0.5 & 0.5\\
        0.3 & 0.7
    \end{pmatrix}
\end{equation*}
\begin{equation*}
    \begin{pmatrix}
        \pi_1 & \pi_2
    \end{pmatrix} \cdot \begin{pmatrix}
        0.5 & 0.5\\
        0.3 & 0.7
    \end{pmatrix} = \begin{pmatrix}
        \pi_1 & \pi_2
    \end{pmatrix}
\end{equation*}

Da \(\sum \pi_i=1\) gilt

\begin{equation*}
    \pi_1 + \pi_2 = 1 \Leftrightarrow \underline{\pi_1 = 1 - \pi_2} \Leftrightarrow \underline{\pi_2 = 1 - \pi_1}
\end{equation*}

\(\pi_1\) und \(\pi_2\) in die folgenden Gleichungen einsetzen:

\begin{equation*}
    \pi_1 \cdot 0.5 + \pi_2 \cdot 0.3 = \pi_1 \Longrightarrow \pi_2 = \frac{5}{8}
\end{equation*}
\begin{equation*}
    \pi_1 \cdot 0.5 + \pi_2 \cdot 0.7 = \pi_2 \Longrightarrow \pi_1 = \frac{3}{8}
\end{equation*}

Daraus folgt der stationäre Zustandsvektor: \(\pi = \begin{pmatrix}
    \frac{3}{8} & \frac{5}{8}
\end{pmatrix}\)



\subsection{Irreduzibilität}

\subsection{Aperiodizität}

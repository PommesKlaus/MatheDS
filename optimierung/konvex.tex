\subsection{Konvexe Funktionen/Mengen}


Eine \emph{Menge} $G \subseteq \mathbb{R}^n$ heißt \emph{konvex}, wenn für alle $x, y \in G$ und $t \in [0, 1]$ gilt:
\begin{equation*}
t x + (1 - t) y \in G
\end{equation*}
D.h. die Verbindungsstrecke zwischen zwei Punkten der Menge liegt komplett in der Menge.
\\
\\
Eine \emph{Funktion} $f: \mathbb{R}^n \rightarrow \mathbb{R}$ heißt \emph{konvex} (\(\leq\)) bzw. \emph{strikt konvex} (\(<\)), wenn für alle $x, y \in \mathbb{R}^n$ und $t \in [0, 1]$ gilt:
\begin{equation*}
    f(t x + (1 - t) y) \leq t f(x) + (1 - t) f(y)
\end{equation*}

\underline{Vorgehensweise:}

\begin{enumerate}
    \item Hesse-Matrix (\(H_f(x)\), =symmetrisch) bestimmen:
            \(H_f(x) =
                \begin{pmatrix}
                    f_{xx} & f_{xy} \\
                    f_{yx} & f_{yy}
                \end{pmatrix}
            \)
    \item Definitheit bestimmen:
        \begin{enumerate}
            \item über \emph{Eigenwerte}
                \begin{itemize}
                    \item Nullstellen charakteristisches Polynom bestimmen (\(\rightarrow \) \ref{sec:charPolynom})
                    \item Interpretation:
                    \begin{itemize}
                        \item alle \(\lambda>0\rightarrow H_f(x)=\) pos. definit
                        \item alle \(\lambda\geq 0\rightarrow H_f(x)=\) pos. semidefinit 
                        \item alle \(\lambda<0\rightarrow H_f(x)=\) neg. definit 
                        \item alle \(\lambda\leq 0\rightarrow H_f(x)=\) neg. semidefinit 
                        \item \(\lambda\) positiv und negativ \(\rightarrow H_f(x)=\) indefinit
                    \end{itemize}
                \end{itemize}
            \item über \emph{Diagonaldominanz}: Ist \(H_f(x)\) Diagonaldominant und alle Diagonalelemente \(>0\), so ist \(H_f(x)\) positiv definit.\\
                \(\rightarrow \) für alle Zeilen: \(||\)Diagonalelement\(|| > \sum ||\)übrige Zeilenelemente\(||\) 
            \item \emph{Choleskyzerlegung} ist möglich \(= H_f(x)\) ist positiv definit
        \end{enumerate}
    \item Konvexität bestimmen:
        \begin{itemize}
            \item \(H_f(x)\) positiv definit \(\Leftrightarrow\) \(f\) strikt konvex
            \item \(H_f(x)\) positiv semidefinit \(\Leftrightarrow\) \(f\) konvex
            \item \(H_f(x)\) negativ definit \(\Leftrightarrow\) \(f\) strikt konvex
            \item \(H_f(x)\) negativ semidefinit \(\Leftrightarrow\) \(f\) konvex
        \end{itemize}
\end{enumerate}
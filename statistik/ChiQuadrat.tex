\subsection{\(\chi^2\)-Test}

\subsubsection{Anpassungstest}

Vergleicht die beobachtete Verteilung einer Stichprobe mit 
einer theoretischen (erwarteten) Verteilung. Es wird geprüft, 
ob die beobachtete Häufigkeitsverteilung von Kategorien mit der 
erwarteten Häufigkeitsverteilung übereinstimmt.\\

\textbf{1. Voraussetzung:} Zufallsvariable \(X\) (z.B. Anzahl der Morde) mit \(s\) Ausprägungen 
(Kategorien; z.B. Wochentage) und \(N\) Beobachtungen (Stichprobenumfang).\\

\textbf{2. Nullhypothese (\(H_0\)):} Die tatsächliche Verteilung entspricht der erwarteten Verteilung (\(P(X \in A_i)=\rho_i\))\\

\textbf{3. Gegenhypothese (\(H_1\)):} Nicht \(H_0\)\\

\textbf{4. \(\chi^2\)-Teststatistik:}

\begin{equation*}
    D_{\rho} = \sum_{i=1}^{s}\frac{(h(i)-N\rho(i))^2}{N\rho(i)}=N\sum_{i=1}^{s}\frac{L(i)^2}{\rho(i)} - N
\end{equation*}

Summe über alle Kategorien \(A_i\) (\(i=1,2,...s\)): 
Differenz aus der Anzahl beobachtete Werte \(h(i)\) und der 
erwarteten Anzahl \(N\rho(i)\) (Multiplikation der 
Wahrscheinlichkeit \(\rho(i)\) mit der Stichprobengröße 
\(N\)). Diese Differenz wird quadriert und durch die erwartete 
Anzahl \(N\rho(i)\) dividiert. Rechte Formelvariante berechnet
die Teststatistik anhand der relativen beobachteten Häufigkeiten \(L(i)\).\\

\textbf{5. Ablehnungsbereich:} \(H_0\) ablehnen, wenn:  \(D_{\rho} > \chi^2_{s-1;1-\alpha}\)\\


\subsubsection{Unabhängigkeitstest}

Prüft die Unabhängigkeit zweiter Merkmale, d.h. ob das Vorkommen einer Variable von der anderen ahängt oder nicht.
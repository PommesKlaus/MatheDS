\subsection{\(\chi^2\)-Test}

\subsubsection{Anpassungstest}

Vergleicht die beobachtete Verteilung einer Stichprobe mit 
einer theoretischen (erwarteten) Verteilung. Es wird geprüft, 
ob die beobachtete Häufigkeitsverteilung von Kategorien mit der 
erwarteten Häufigkeitsverteilung übereinstimmt.\\

\textbf{1. Voraussetzung:} Zufallsvariable \(X\) (z.B. Ergebnis eines Würfelwurfs) mit \(s\) Ausprägungen 
(Kategorien; z.B. 1-6 Würfelaugen) und \(N\) Beobachtungen (Stichprobenumfang).\\

\textbf{2. Nullhypothese (\(H_0\)):} Die tatsächliche Verteilung entspricht der erwarteten Verteilung (\(P(X \in A_i)=\rho_i=\frac{1}{6}\))\\

\textbf{3. Gegenhypothese (\(H_1\)):} Nicht \(H_0\)\\

\textbf{4. \(\chi^2\)-Teststatistik:}

\begin{equation*}
    D_{\rho} = \sum_{i=1}^{s}\frac{(h(i)-N\rho(i))^2}{N\rho(i)}=\left(\sum_{i=1}^{s}\frac{h(i)^2}{N\rho(i)}\right)-N=N\left(\sum_{i=1}^{s}\frac{L(i)^2}{\rho(i)}\right) - N
\end{equation*}
wobei
\begin{itemize}
    \item \(N\) = Stichprobengröße \emph{(z.B. 50 Würfe mit Würfel)}
    \item \(s\) = Anzahl der Kategorien \emph{(z.B. 6 Würfelaugen)}
    \item \(A_i\) = Kategorie \(i\) \emph{(z.B. Würfelaugen 1-6)}
    \item \(h(i)\) = Anzahl der Beobacht. in Kategorie \(A_i\) \emph{(z.B. 8 Würfe 1er Würfe)}
    \item \(L(i)=\frac{h(i)}{N}\) = relative Häufigkeit der Beobachtungen in Kategorie \(A_i\) \emph{(z.B. 8/50 Würfe mit Würfelaugen 1 usw.)}
    \item \(\rho(i)\) = Wahrscheinlichkeit der Kategorie \(A_i\) \emph{(z.B. \(\frac{1}{6}\) für Würfelauge 1 usw.)}
    \item \(\alpha\) = Irrtumswahrscheinlichkeit/Signifikanzniveau \emph{(z.B. 5\%)}
\end{itemize}

\textbf{5. Ablehnungsbereich:} \(H_0\) ablehnen, wenn:  \(D_{\rho} > \chi^2_{s-1;1-\alpha}\)\\


\subsubsection{Unabhängigkeitstest}

Prüft die Unabhängigkeit zweiter Merkmale, d.h. ob das Vorkommen einer Variable von der anderen ahängt oder nicht.\\


\textbf{1. Voraussetzung:} Zufallsvariable \(X\) und \(Y\) nehmen jeweils zwei Werte an (z.B. \(X\)=männlich/weiblich, \(Y\)=raucht/nicht raucht).\\

\textbf{2. Nullhypothese (\(H_0\)):} \(X\) und \(Y\) sind stochastisch unabhängig\\

\textbf{3. Gegenhypothese (\(H_1\)):} Nicht \(H_0\)\\

\textbf{4. \(\chi^2\)-Teststatistik:}\\

Aufstellen einer Vierfeldertafel:\\

\begin{minipage}[c]{.4\textwidth}
    \centering
    \begin{tabular}{c|c|c|c}
                   & $Y$           & $\overline{Y}$ & $\Sigma$         \\ 
                   \hline
    $X$            & \(N_{11}\)    & \(N_{12}\)     & \(N_{1 \cdot} \) \\ 
    \hline
    $\overline{X}$ & \(N_{21}\)    & \(N_{22}\)     & \(N_{2\cdot} \)  \\ 
    \hline
    $\Sigma$       & $N_{\cdot 1}$ & $N_{\cdot 2}$  & $n$
    \end{tabular}
\end{minipage}
\begin{minipage}[c]{.6\textwidth}
    \centering
    \begin{tabular}{c|c|c|c}
             & Nichtraucher & Raucher & $\Sigma$ \\ 
    \hline
    männlich & 170          & 30      & 200 \\
    \hline
    weiblich & 250          & 150     & 400  \\
    \hline
    $\Sigma$ & 420          & 180     & 600
    \end{tabular}
\end{minipage}\\\\

Daraus Berechnung der Teststatistik:
\begin{equation*}
    D_{\rho} = n \frac{(N_{11}N_{22}-N_{12}N_{21})^2}{N_{1\cdot}N_{2\cdot}N_{\cdot1}N_{\cdot2}}
\end{equation*}

\emph{Gedankenstütze: Determinante hoch 2 geteilt durch Produkt aus allen Spalten- und Zeilensummen}\\

\textbf{5. Ablehnungsbereich:} \(H_0\) ablehnen, wenn:  \(T > \chi^2_{1;1-\alpha}\)\\
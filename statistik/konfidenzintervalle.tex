\subsection{Konfidenzintervalle}

Schätzung eines Intervalls, in dem sich der wahre Wert (z.B. der Erwartungswert~\(\mu\)) mit einer gewissen Wahrscheinlichkeit befindet.

\subsubsection{Vorgehensweise bei Normalverteilung und \textbf{bekanntem} \boldmath{\(\sigma\)}}

\begin{enumerate}
    \item Punktschätzung des Erwartungswerts aus \(n\) Stichproben (\(x_i\))
            \[M(x)=\frac{1}{n}\sum_{i=1}^{n}x_i\]\\
            \(\rightarrow\) dieser entspricht i.d.R. nicht dem wahren Wert \(\mu\) der Grundgesamtheit.

    \item Konfidenzniveau (\(1-\alpha\); \(\alpha=\) Irrtumsniveau) festlegen und \(z_{1-\frac{\alpha}{2}}\) aus Tabelle zur Normalverteilung ablesen
        \begin{itemize}
            \item \(90\% = 1-\alpha \rightarrow \alpha=0.1\rightarrow z_{0.95}=1.65\)
            \item \(95\% \rightarrow \alpha=0.05\rightarrow z_{0.975}=1.96\)
            \item \(96\% \rightarrow \alpha=0.04\rightarrow z_{0.980}=2.06\)
            \item \(97\% \rightarrow \alpha=0.03\rightarrow z_{0.985}=2.17\)
            \item \(98\% \rightarrow \alpha=0.02\rightarrow z_{0.99}=2.33\)
            \item \(99\% \rightarrow \alpha=0.01\rightarrow z_{0.995}=2.58\)
        \end{itemize}
    \item Berechnung des Konfidenzintervalls
        \begin{equation*}
            \mathcal{I}(x) = \left[M(x) - z_{1-\frac{\alpha}{2}} \cdot \frac{\sigma}{\sqrt{n}}\hspace{0.5em}; \hspace{0.5em}M(x) + z_{1-\frac{\alpha}{2}} \cdot \frac{\sigma}{\sqrt{n}}\right]
        \end{equation*}
        \emph{ist ein Konfidenzintervall zum Sicherheitsniveau \(1-\alpha\)}.\\\\
        \underline{\emph{Vorgehensweise:}} \(1-\frac{\alpha}{2}\) berechnen und innerhalb der Tabelle zur Normalverteilung diesen Wert suchen. Der gesuchte \(z\)-Wert ergibt sich dann aus den Zeilen- und Spaltenberschriften.\\
\end{enumerate}

\subsubsection{Vorgehensweise bei Normalverteilung und \textbf{unbekanntem} \boldmath{\(\sigma\)}}

Grds. analog zu oben, wobei Werte für \(t_{n-1,1-\frac{\alpha}{2}}\) aus der t-Verteilung verwendet werden (Studentsche (\(t\)-) Verteilung mit \(n-1\) Freiheitsgraden).  Nur bei \(n\geq 30\).

\begin{enumerate}
    \item Punktschätzung des Erwartungswerts aus \(n\) Stichproben (\(x_i\))
            \[M(x)=\frac{1}{n}\sum_{i=1}^{n}x_i\]\\
            \(\rightarrow\) dieser entspricht i.d.R. nicht dem wahren Wert \(\mu\) der Grundgesamtheit.
    \item Berechnung des Standardfehlers der Stichprobenmittelwerte
        \[\text{korrigierte Stichprobenvarianz: }V^*(x)=\frac{1}{n-1}\sum_{i=1}^{n}(x_i-M(x))^2\]
        \[\text{Standardfehler: }s^*=\sqrt{\frac{V^*(x)}{n}}\]
        \(\rightarrow\) dieser entspricht i.d.R. nicht dem wahren Wert \(\sigma\) der Grundgesamtheit.
    \item Berechnung des Konfidenzintervalls
    \begin{equation*}
        \mathcal{I}(x) = \left[M(x) - t_{n-1;1-\frac{\alpha}{2}} \cdot s^*\hspace{0.5em}; \hspace{0.5em}M(x) + t_{n-1;1-\frac{\alpha}{2}} \cdot s
        ^*\right]
    \end{equation*}
    \emph{ist ein Konfidenzintervall zum Sicherheitsniveau \(1-\alpha\) mit \(n-1\) Freiheitsgraden}\\\\
    \underline{\emph{Vorgehensweise:}} Richtige Zeile für Freiheitsgrade \(n-1\) suchen. In Spalte \(1-\frac{\alpha}{2}\) suchen. Der gewünschte \(t\)-Wert ergibt sich aus der Tabelle.\\
\end{enumerate}

\subsubsection{Vorgehensweise im Binominalmodell}

Wenn X binominalverteilt ist (\(X \sim  B(n, p)\), \(n\)=Anzahl gezogene Versuche, \(p\)=Erfolgswahrscheinlichkeit), \(n\) groß und die Varianz nicht zu klein ist (Faustregel: \(np(1-p) > 9\)), gilt die Approximation durch die Normalverteilung mit:

\begin{itemize}
    \item Erwartungswert: \(\mu=np\)
    \item Standardabweichung: \(\sigma=\sqrt{np(1-p)}\)
    \item Standardfehler: \(s=\sqrt{\frac{p(1-p)}{n}}\)
    \item Punktschätzung: \(\hat{p}=\frac{x}{n}\)
    \item Konfidenzintervall für \(p\): \(\mathcal{I}(p)\approx \left[\hat{p}-z_{1-\frac{\alpha}{2}}\cdot s; \hat{p}+z_{1-\frac{\alpha}{2}}\cdot s\right]\)
    \item Konfidenzintervall für \(\mu\): \(\mathcal{I}(\mu)\approx \left[\hat{\mu}-z_{1-\frac{\alpha}{2}}\cdot s; \hat{\mu}+z_{1-\frac{\alpha}{2}}\cdot s\right]\)
\end{itemize}

\underline{\emph{Vorgehensweise:}} \(1-\frac{\alpha}{2}\) berechnen und innerhalb der Tabelle zur Normalverteilung diesen Wert suchen. Der gesuchte \(z\)-Wert ergibt sich dann aus den Zeilen- und Spaltenberschriften.\\
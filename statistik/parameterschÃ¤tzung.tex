\subsection{Parameterschätzung}

Kriterien für gute Schätzer:
\begin{itemize}
    \item \textbf{Konsistenz:} Schätzungen werden genauer, je größer die Stichprobe ist
    \begin{itemize}
        \item Ein Schätzer \(T\) ist konsistent für \(\theta\), wenn \(\lim_{n\rightarrow\infty}P(|T-\theta|>\varepsilon)\rightarrow 0\) bzw. \(\lim_{n\rightarrow\infty}Var(T)\rightarrow 0\)
    \end{itemize}
    \item \textbf{Erwartungstreue/Unverzerrtheit/unbiased:} Schätzer liegt im Mittel richtig
    \begin{textblock*}{5cm}(11.5cm,6.2cm) % {block width} (coords) 
        \footnotesize{Bei gegebenem Schätzer \(T(x)=x^2-x\) und Verteilung \(\mathbb{P}_\Theta(\{n\})=\frac{\Theta^n}{n!e^\Theta}\)}
     \end{textblock*}
    \begin{itemize}
        \item Ein Schätzer \(T\) ist erwartungstreu für \(\theta\), wenn \(E(T)=\theta\); z.B. \(E(T)=E(T(X))=\sum_{k=0}^{\infty}\underbrace{T(k)\cdot\text{Wkt.fn}}_{(k^2-k)\frac{\Theta^k}{k!e^\Theta}}\)
        \item z.B. \(\hat{\mu}=\frac{1}{n}\sum_{i=1}^{n}x_i\) liegt im Mittel bei \(\mu\). 
        \item \underline{Achtung:} Stichprobenvarianz \(s^2=\frac{1}{n}\sum_{i=1}^{n}(x_i-\hat{\mu})^2\) ist \underline{nicht} erwartungstreu, da \(\hat{\mu}\) in \(s^2\) eingeht, Ausreißer landen systematisch seltener in der Stichprobe, so dass \(s^2\) zu klein ist. Lösung: Korrektur, so dass Schätzer der Stichprobenvarianz \(\hat{\sigma}^2=\mathbf{\frac{n}{n-1}}\cdot\frac{1}{n}\sum_{i=1}^{n}(x_i-\hat{\mu})^2\)
        \item \underline{Beispiel}: Stichprobe mit \(N=5\text{, }s^2=2 \text{ und } \bar{x}=32\) \(\rightarrow\) Schätzer Varianz \(\hat{\sigma}^2=\frac{5}{5-1}\cdot 2=2.5\) und Standardschätzfehler \(\hat{\sigma}_{\bar{x}}=\sqrt{\frac{\hat{\sigma}^2}{n}}=\sqrt{\frac{2.5}{5}}=0.707\)
    \end{itemize}
    \item \emph{Beispiel 1:} \(T_1=\bar{X}\text{ mit }X_1, X_2, \cdots, X_n\sim N(\mu, \sigma)\)
    \begin{itemize}
        \item \(E(T_1)=E(\bar{X})=E(\frac{1}{n}\sum_{i=1}^{n}X_i)=\frac{1}{n}\sum_{i=1}^{n}E(X_i)=\frac{1}{n}\sum_{i=1}^{n}\mu=\frac{1}{n}n\mu=\mu \rightarrow\) erwartungstreu
        \item \(Var(T_1)=Var(\bar{X})=Var(\frac{1}{n}\sum_{i=1}^{n}X_i)=\frac{1}{n^2}\sum_{i=1}^{n}Var(X_i)=\frac{1}{n^2}\sum_{i=1}^{n}\sigma^2=\frac{1}{n^2}n\sigma^2=\frac{\sigma^2}{n} \rightarrow 0\) für \(\lim_{n\rightarrow\infty}\rightarrow\) konsistent
    \end{itemize} 
    \item \emph{Beispiel 2:} \(T_2=\frac{1}{2}(X_1+X_n)\text{ mit }X_1, X_2, \cdots, X_n\sim N(\mu, \sigma)\)
    \begin{itemize}
        \item \(E(T_2)=E(\frac{1}{2}(X_1+X_n))=\frac{1}{2}(E(X_1)+E(X_n))=\frac{1}{2}(\mu+\mu)=\mu \rightarrow\) erwartungstreu
        \item \(Var(T_2)=Var(\frac{1}{2}(X_1+X_n))=\left(\frac{1}{2}\right)^2(Var(X_1)+Var(X_n))=\frac{1}{4}(\sigma^2+\sigma^2)=\frac{\sigma^2}{2} \nrightarrow  0 \rightarrow\) nicht konsistent 
    \end{itemize}
\end{itemize}

\textbf{Zentrale Größen:}
\begin{itemize}
    \item Kovarianz: \(cov(X,Y)=E[(X-E[X])(Y-E[Y])]=E(XY)-E(X)E(Y)\)
    \item Korrelation: \(corr(X,Y)=\frac{cov(X,Y)}{\sigma_X\sigma_Y}\) (liegt zwischen -1 und +1)
    \item Standardschätzfehler des Mittelwerts: \(\hat{\sigma}_{\bar{x}}=\sqrt{\frac{\hat{\sigma}^2}{n}}\)
\end{itemize}


\subsubsection{Maximum-Likelihood Schätzer (ML-Schätzer)}

\textbf{Wichtige ML-Schätzer:}

\begin{itemize}
    \item \textbf{\emph{Binomialverteilung}} (\(X\sim Bin(n, \pi)\), \(n\)=Länge der Versuchsreihe, \(\pi\)=Wahrscheinlichkeit für Erfolg):
    \begin{itemize}
        \item Erwartungswert \(\hat{\pi}=T(x)=\frac{x}{n} \rightarrow \) Anzahl Erfolge / Anzahl Versuche
        \item Varianz \(\hat{\sigma}^2=\frac{\pi(1-\pi)}{n} \rightarrow \) ggf. mit \(\hat{\pi}\) rechnen
    \end{itemize}
    \item \textbf{\emph{Normalverteilung}} (\(X\sim N(\mu, \sigma^2)\), \(\mu\)=Erwartungswert, \(\sigma^2\)=Varianz):
    \begin{itemize}
        \item Erwartungswert \(\hat{\mu}=T(x)=\frac{1}{n}\sum_{i=1}^{n}x_i \rightarrow \) arithmetisches Mittel
        \item Varianz \(S^2=\frac{1}{n}\sum_{i=1}^{n}(x_i-\hat{\mu})^2 \rightarrow \) empirische Varianz
        \item korrigierte Stichprobenvarianz \(S^{*2}=\frac{1}{n-1}\sum_{i=1}^{n}(x_i-\hat{\mu})^2 \)
    \end{itemize}
\end{itemize}

\textbf{Mathematische Bestimmung eines ML-Schätzers:}

\begin{enumerate}
    \item Aufstellen der ML-Funktion: \(L(\theta)=\prod_{i=1}^{n}f(x_i)\); mit Dichtefunktion \(f(x_i)\) (vgl. S.~\pageref{verteilungen})
    \item Logarithmierung: \(\ln(L(\theta))=\sum_{i=1}^{n}\ln(f(x_i))\); Umformen/Vereinfachen mit Logarithmengesetzen (S.~\pageref{logarithmusgesetze})
    \item Ableiten nach \(\theta\) und Nullsetzen: \(\frac{\partial \ln(L(\theta))}{\partial \theta}=0\)
    \item Lösen der Gleichung nach \(\theta\)
    \item Überprüfen, ob es sich um ein Maximum handelt
\end{enumerate}

\textbf{Beispiel 1 für Herleitung einer Dichtefunktion:}

Bestimme \(\theta\) mit der ML-Methode für die Dichtefunktion\\

\(f(x) = \begin{cases}
    4x^3\theta e^{-\theta x^4} & \text{für } x>0, \theta>0\\
    0 & \text{sonst}
\end{cases}\)\\

\begin{enumerate}
    \item Aufstellen der ML-Funktion: \(L(\theta)=\prod_{i=1}^{n}f(x_i)=\prod_{i=1}^{n}4x_i^3\theta e^{-\theta x_i^4}=\overbrace{4^n(x_1^3x_2^3\hdots x_n^3)\theta^n e^{-\theta(x_1^4+x_2^4+\hdots+x_n^4)}}^{\text{Sinnvoll zusammenfassen}}\)
    \item Logarithmierung: \(\ln(L(\theta))=\ln(4^n(x_1^3x_2^3\hdots x_n^3))+n\ln(\theta)-\theta(x_1^4+x_2^4+\hdots+x_n^4)\underbrace{\ln(e)}_{=1}\)
    \item Ableiten nach \(\theta\): \(\frac{\partial \ln(L(\theta))}{\partial \theta}=n\frac{1}{\theta}-\sum_{i=1}^{n}x_i^4\stackrel{!}{=}0\Leftrightarrow \theta=\frac{n}{\sum_{i=1}^{n}x_i^4}\)
\end{enumerate}



\textbf{Beispiel 2 für Herleitung (\(X\sim N(\mu, \sigma)\)):}

\begin{enumerate}
    \item Aufstellen der ML-Funktion: \(L(\mu, \sigma^2)=\prod_{i=1}^{n}f(x_i)=\prod_{i=1}^{n}\frac{1}{\sqrt{2\pi\sigma^2}}e^{-\frac{(x_i-\mu)^2}{2\sigma^2}}\)
    \item Logarithmierung:\\
        \begin{equation*}
            \begin{split}
            \ln(L(\mu, \sigma^2))&=\sum_{i=1}^{n}\ln(\frac{1}{\sqrt{2\pi\sigma^2}}e^{-\frac{(x_i-\mu)^2}{2\sigma^2}})\\
            &=\sum_{i=1}^{n}\left [ \ln(\frac{1}{\sqrt{2\pi\sigma^2}})-\frac{(x_i-\mu)^2}{2\sigma^2}\right ]\\
            &=-\frac{n}{2}\ln(2\pi\sigma^2)-\frac{1}{2\sigma^2}\sum_{i=1}^{n}(x_i-\mu)^2\\
            \end{split}
        \end{equation*}
    \item Ableiten nach \(\mu\) und \(\sigma^2\) und Nullsetzen:
        \begin{equation*}
            \begin{split}
            \frac{\partial \ln(L(\mu, \sigma^2))}{\partial \mu}&=\frac{1}{\sigma^2}\sum_{i=1}^{n}(x_i-\mu)=0\\
            % &\Rightarrow \sum_{i=1}^{n}(x_i-\mu)=0\\
            &\Leftrightarrow \sum_{i=1}^{n}x_i-\sum_{i=1}^{n}\mu=0\\
            % &\Rightarrow \sum_{i=1}^{n}x_i-n\mu=0\\
            &\Leftrightarrow \sum_{i=1}^{n}x_i=n\mu \Leftrightarrow \mu=\frac{1}{n}\sum_{i=1}^{n}x_i\\
        %     \end{split}
        % \end{equation*}
        % \begin{equation*}
        %     \begin{split}
            \frac{\partial \ln(L(\mu, \sigma^2))}{\partial \sigma^2}&=-\frac{n}{2\sigma^2}+\frac{1}{2(\sigma^2)^2}\sum_{i=1}^{n}(x_i-\mu)^2=0\\
            &\Leftrightarrow \frac{n}{2\sigma^2}=\frac{1}{2(\sigma^2)^2}\sum_{i=1}^{n}(x_i-\mu)^2\\
            &\Leftrightarrow \frac{n}{2}=\frac{1}{2\sigma^2}\sum_{i=1}^{n}(x_i-\mu)^2\\
            &\Leftrightarrow \sigma^2=\frac{1}{n}\sum_{i=1}^{n}(x_i-\mu)^2 \Leftrightarrow \sigma=\sqrt{\frac{1}{n}\sum_{i=1}^{n}(x_i-\mu)^2}\\
            \end{split}
        \end{equation*}
\end{enumerate}

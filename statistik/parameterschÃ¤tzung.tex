\subsection{Parameterschätzung}

Zentrale Größen:
\begin{itemize}
    \item Kovarianz: \(cov(X,Y)=E[(X-E[X])(Y-E[Y])]=E(XY)-E(X)E(Y)\)
    \item Korrelation: \(corr(X,Y)=\frac{cov(X,Y)}{\sigma_X\sigma_Y}\) (liegt zwischen -1 und +1)
\end{itemize}

Kriterien für gute Schätzer:
\begin{itemize}
    \item \textbf{Konsistenz:} Schätzungen werden genauer, je größer die Stichprobe ist
    \item \textbf{Erwartungstreue/Unverzerrtheit/unbiased:} Schätzer liegt im Mittel richtig.
\end{itemize}

\subsubsection{Maximum-Likelihood Schätzer (ML-Schätzer)}

\begin{itemize}
    \item \textbf{Binomialverteilung} (\(X\sim Bin(n, \pi)\), \(n\)=Länge der Versuchsreihe, \(\pi\)=Wahrscheinlichkeit für Erfolg):
    \begin{itemize}
        \item Erwartungswert \(\hat{\pi}=T(x)=\frac{x}{n} \rightarrow \) Anzahl Erfolge / Anzahl Versuche
        \item Varianz \(\hat{\sigma}^2=\frac{\pi(1-\pi)}{n} \rightarrow \) ggf. mit \(\hat{\pi}\) rechnen
    \end{itemize}
    \item \textbf{Normalverteilung} (\(X\sim N(\mu, \sigma^2)\), \(\mu\)=Erwartungswert, \(\sigma^2\)=Varianz):
    \begin{itemize}
        \item Erwartungswert \(\hat{\mu}=T(x)=\frac{1}{n}\sum_{i=1}^{n}x_i \rightarrow \) arithmetisches Mittel
        \item Varianz \(S^2=\frac{1}{n}\sum_{i=1}^{n}(x_i-\hat{\mu})^2 \rightarrow \) empirische Varianz
        \item korrigierte Stichprobenvarianz \(S^{*2}=\frac{1}{n-1}\sum_{i=1}^{n}(x_i-\hat{\mu})^2 \)
    \end{itemize}
\end{itemize}

\textbf{\emph{Beispiel für Herleitung (\(X\sim N(\mu, \sigma)\)):}}

\begin{enumerate}
    \item Aufstellen der ML-Funktion: \(L(\mu, \sigma^2)=\prod_{i=1}^{n}f(x_i)=\prod_{i=1}^{n}\frac{1}{\sqrt{2\pi\sigma^2}}e^{-\frac{(x_i-\mu)^2}{2\sigma^2}}\)
    \item Logarithmierung:\\
        \begin{equation*}
            \begin{split}
            \ln(L(\mu, \sigma^2))&=\sum_{i=1}^{n}\ln(\frac{1}{\sqrt{2\pi\sigma^2}}e^{-\frac{(x_i-\mu)^2}{2\sigma^2}})\\
            &=\sum_{i=1}^{n}\left [ \ln(\frac{1}{\sqrt{2\pi\sigma^2}})-\frac{(x_i-\mu)^2}{2\sigma^2}\right ]\\
            &=-\frac{n}{2}\ln(2\pi\sigma^2)-\frac{1}{2\sigma^2}\sum_{i=1}^{n}(x_i-\mu)^2\\
            \end{split}
        \end{equation*}
    \item Ableiten nach \(\mu\) und \(\sigma^2\) und Nullsetzen:\\
        \begin{equation*}
            \begin{split}
            \frac{\partial \ln(L(\mu, \sigma^2))}{\partial \mu}&=\frac{1}{\sigma^2}\sum_{i=1}^{n}(x_i-\mu)=0\\
            &\Rightarrow \sum_{i=1}^{n}(x_i-\mu)=0\\
            &\Rightarrow \sum_{i=1}^{n}x_i-\sum_{i=1}^{n}\mu=0\\
            &\Rightarrow \sum_{i=1}^{n}x_i-n\mu=0\\
            &\Rightarrow \sum_{i=1}^{n}x_i=n\mu\\
            &\Rightarrow \mu=\frac{1}{n}\sum_{i=1}^{n}x_i\\
            \frac{\partial \ln(L(\mu, \sigma^2))}{\partial \sigma^2}&=-\frac{n}{2\sigma^2}+\frac{1}{2(\sigma^2)^2}\sum_{i=1}^{n}(x_i-\mu)^2=0\\
            &\Rightarrow \frac{n}{2\sigma^2}=\frac{1}{2(\sigma^2)^2}\sum_{i=1}^{n}(x_i-\mu)^2\\
            &\Rightarrow \frac{n}{2}=\frac{1}{2\sigma^2}\sum_{i=1}^{n}(x_i-\mu)^2\\
            &\Rightarrow \sigma^2=\frac{1}{n}\sum_{i=1}^{n}(x_i-\mu)^2\\
            &\Rightarrow \sigma=\sqrt{\frac{1}{n}\sum_{i=1}^{n}(x_i-\mu)^2}\\
            \end{split}
        \end{equation*}
\end{enumerate}

\subsubsection{Bayes-Schätzer}
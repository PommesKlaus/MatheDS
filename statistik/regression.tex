\subsection{Zentraler Grenzwertsatz}

Der zentrale Grenzwertsatz besagt, dass die Summe von unabhängigen, identisch verteilten Zufallsvariablen einer beliebigen Verteilung für \(n\rightarrow\infty\) gegen eine Normalverteilung konvergiert, auch wenn die Ausgangsverteilung nicht normalverteilt ist. Als Theorem mit \underline{E}rwartungswert (\(\mu\)), \underline{V}arianz (\(\sigma^2\)) und \(n\) Zufallsvariablen:\\


\begin{equation*}
    \lim_{n\rightarrow\infty} P\left(\frac{\sum_{i=1}^{n}X_i - n\mu}{\sigma\sqrt{n}}\leq x\right) = \sqrt{\frac{n}{V}}\left(\frac{1}{n}\sum_{i=1}^{n}X_i-E\right) \rightarrow \Phi(x) \sim N(0,1)   
\end{equation*}

\subsection{Regression}

Unterscheidung zwischen (einfacher/multipler) linearer und logistischer Regression:

\begin{itemize}
    \item \textbf{Lineare Regression:}
    \begin{itemize}
        \item Ziel: Schätzung des Erwartungswerts einer abhängigen Variable \(Y\) in Abhängigkeit von einer oder mehreren unabhängigen Variablen \(X\)
        \item Modell: \(Y = \beta_0 + \beta_1 \cdot X_1 + \beta_2 \cdot X_2 + \ldots + \beta_n \cdot X_n + \varepsilon\)
        \item \(\varepsilon\): Fehlerterm
        \item \(\beta_0\): Achsenabschnitt
        \item \(\beta_1, \beta_2, \ldots, \beta_n\): Regressionskoeffizienten
    \end{itemize}
    \item \textbf{Logistische Regression:}
    \begin{itemize}
        \item Ziel: Schätzung der Wahrscheinlichkeit, dass eine abhängige Variable \(Y\) den Wert 1 annimmt, in Abhängigkeit von einer oder mehreren unabhängigen Variablen \(X\)
        \item Modell: \(P(Y=1) = \frac{1}{1+e^{-(\beta_0 + \beta_1 \cdot X_1 + \beta_2 \cdot X_2 + \ldots + \beta_n \cdot X_n)}}\)
        \item \(\beta_0, \beta_1, \beta_2, \ldots, \beta_n\): Regressionskoeffizienten
    \end{itemize}
\end{itemize}
\subsection{Verteilungen}

\subsubsection{Binomialverteilung (\(X\sim Bin(n,\pi)\))}
Diskrete Wahrscheinlichkeitsverteilung; beschreibt die Anzahl an Erfolgen in einer Serie von unabhängigen Versuchen, die jeweils genau zwei Ergebnisse haben (Erfolg/Misserfolg); z.B. beim Münzwurf, Ziehen mit Zurücklegen.

\begin{itemize}
    \item \(n\)=Anzahl der Versuche/Ziehungen, \(\pi\)=Erfolgswahrscheinlichkeit (z.B. \(0.3\))
    \item \textbf{Wahrscheinlichkeit:} \(P(X=k)=\binom{n}{k}\pi^k(1-\pi)^{n-k}\); \(k=\) Anzahl der gewünschten Erfolge
    \item \textbf{Erwartungswert:} \(E(X)=n\pi\)
    \item \textbf{Varianz:} \(Var(X)=n\pi(1-\pi)\)
    \item \textbf{Normalapproximation:} \(X\sim N(n\pi, n\pi(1-\pi))\)
    \item \textbf{Binomialkoeffizient:} \(\binom{n}{k}=\frac{n!}{k!(n-k)!}\)
    \item \textbf{Bernoulli-Verteilung:} Spezialfall der Binomialverteilung mit \(n=1\)
\end{itemize}


\subsubsection{Poisson-Verteilung (\(X\sim Poi(\mu)\))}
Diskrete Wahrscheinlichkeitsverteilung; beschreibt die Anzahl an Ereignissen in einem festgelegten Zeitintervall, wenn die Ereignisse mit einer konstanten Rate und unabhängig von der Zeit auftreten; z.B. Anzahl der Anrufe in einer Stunde, Anzahl der Kunden in einer Schlange, Anzahl der Fehler in einem Text. Für kleine \(\mu\) zeigt die Poisson-Verteilung eine starke Asysmmetrie (Rechtsschiefe).

\begin{itemize}
    \item \(\varTheta \)=Erwartungswert (z.B. \(0.3\))
    \item \textbf{Wahrscheinlichkeit:} \(P(X=n)=\frac{\varTheta^n}{n!}e^{-\varTheta}\); \(n=\) Anzahl der gewünschten Ereignisse
    \item \textbf{Erwartungswert:} \(E(X)=\varTheta\)
    \item \textbf{Varianz:} \(Var(X)=\varTheta\)
    \item \textbf{Normalapproximation:} \(X\sim N(\varTheta, \varTheta)\)
\end{itemize}


\subsubsection{Hypergeometrische Verteilung (\(X\sim H(n,N,m)\))}

Ähnlich wie Binomialverteilung, aber ohne Zurücklegen; z.B. beim Ziehen ohne Zurücklegen aus einer Urne mit \(N\) Kugeln, davon \(m\) mit Erfolgsmarkierung.

\begin{itemize}
    \item \(n\)=Anzahl der Versuche/Ziehungen, \(N\)=Anzahl der Kugeln in der Urne, \(m\)=Anzahl der Kugeln mit Erfolgsmarkierung
    \item \textbf{Wahrscheinlichkeit:} \(P(X=k)=\frac{\binom{m}{k}\binom{N-m}{n-k}}{\binom{N}{n}}\); \(k=\) Anzahl der gewünschten Erfolge
    \item \textbf{Erwartungswert:} \(E(X)=n\frac{m}{N}\)
    \item \textbf{Varianz:} \(Var(X)=n\frac{m}{N}\left(1-\frac{m}{N}\right)\left(\frac{N-n}{N-1}\right)\)
    \item \textbf{Normalapproximation:} \(X\sim N\left(n\frac{m}{N}, n\frac{m}{N}\left(1-\frac{m}{N}\right)\left(\frac{N-n}{N-1}\right)\right)\)
\end{itemize}


\subsubsection{Normalverteilung (\(X\sim N(\mu, \sigma^2)\))}
Stetige Wahrscheinlichkeitsverteilung; beschreibt viele natürliche Vorgänge (z.B. Körpergröße, IQ, Fehler in Messungen); zentrales Grenzwerttheorem: Summe von unabhängigen Zufallsvariablen strebt gegen Normalverteilung; symmetrisch um \(\mu\), \(\sigma^2\)-bestimmte Breite; \(\mu\)=Erwartungswert, \(\sigma^2\)=Varianz, \(\sigma\)=Standardabweichung.\\

\begin{minipage}{0.5\textwidth}
    \begin{itemize}
        \item \textbf{Erwartungswert:} \(E(X)=\mu\)
        \item \textbf{Varianz:} \(Var(X)=\sigma^2\)
        \item \textbf{Standardnormalverteilung:} \(Z\sim N(0,1)\)
    \end{itemize}
\end{minipage}
\begin{minipage}{0.2\textwidth}
    \underline{Dichtefunktion}:
    \begin{equation*}
        f(x)=\frac{1}{\sqrt{2\pi\sigma^2}}e^{-\frac{(x-\mu)^2}{2\sigma^2}}
    \end{equation*}
\end{minipage}